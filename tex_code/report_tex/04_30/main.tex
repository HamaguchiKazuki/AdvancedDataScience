\documentclass[uplatex,b5j]{jsarticle} % uplatexオプションを入れる
\usepackage[scale=.8]{geometry}      % ページレイアウトを変更してみる
\usepackage[T1]{fontenc}             % T1エンコーディングにしてみる
\usepackage{txfonts}                 % 欧文フォントを変えてみる
\usepackage{plext}                   % pLaTeX付属の縦書き支援パッケージ
\usepackage{okumacro}                % jsclassesに同梱のパッケージ
\usepackage{amsmath}
\begin{document}
\title{データ科学特論 課題1}
\author{2120041 濱口和希}
\西暦\maketitle                      % 漢字のマクロ名もOK

\section{$E_p[X], V_p[X]$の計算}
まず,$E_p[X]$について求める.
問題より
\begin{eqnarray}
    p(x|\theta)=\frac{\theta}{2}\mathrm{e}^{(-\theta|x-\mu|)} \label{eq:1} & \\
    E_p[X]=\int_{-\infty}^{\infty}xp(x|\theta)dx \label{eq:2} & \\
    V_p[X]=\int_{-\infty}^{\infty}(x-E_p[X])^2p(x|\theta)dx & \label{eq:3}
\end{eqnarray}
数式(\ref{eq:2})に数式(\ref{eq:1})を代入し定数を前に出すと
\begin{equation}
    E_p[X]=\frac{\theta}{2}\int_{-\infty}^{\infty}x\mathrm{e}^{(-\theta|x-\mu|)}dx \label{eq:4}
\end{equation}
ここで$|x-\mu|$の範囲は次のように場合分けできる
\begin{equation}\label{eq:5}
    |x-\mu| = \begin{cases}
    -x+\mu & (x<\mu) \\
     x-\mu & (x>\mu)
    \end{cases}
\end{equation}
式(\ref{eq:4})を不定積分として考えると$x<\mu$の時は次のように表すことができる.
\begin{equation}
    E_p[X]=\frac{\theta}{2}\int x\mathrm{e}^{(\theta x -\theta \mu)} dx \label{eq:6}
\end{equation}
被積分関数$x\mathrm{e}^{(\theta x -\theta \mu)}$について,$s=\theta x -\theta \mu, ds=\theta dx$と置換する.
\begin{equation}
    =\frac{1}{2\theta}\int \mathrm{e}^{s}(s+\theta \mu) ds \label{eq:7}
\end{equation}
被積分関数$\mathrm{e}^{s}(s+\theta \mu)$を展開し多項式の積分を各項に分けて定数を前に出す.
\begin{equation}
    =\frac{1}{2\theta}\int \mathrm{e}^{s}s ds + \frac{\mu}{2}\int \mathrm{e}^{s} ds \label{eq:8}
\end{equation}
被積分関数$\mathrm{e}^{s}s ds$について,部分積分を適用し計算する.
\begin{equation}
    =\frac{\mathrm{e}^{s}}{2\theta}(s+\theta\mu-1) \label{eq:9}
\end{equation}
$s=\theta x -\theta \mu$を式(\ref{eq:9})に代入する.
\begin{equation}
    =\frac{(\theta x-1)\mathrm{e}^{\theta(x-\mu)}}{2\theta} + C (Cは積分定数) \label{eq:10}
\end{equation}
$x>\mu$の時も同様に計算する.
\begin{equation}
    -\frac{(\theta x+1)\mathrm{e}^{-\theta(x-\mu)}}{2\theta} + C (Cは積分定数) \label{eq:11}
\end{equation}
任意の大きな数$A$とするとし,式(\ref{eq:10}),式(\ref{eq:11})を使用して,$[-A,\mu][\mu,A]$の区間で定積分を行うと
\begin{eqnarray} \label{eq:12}
    E_p[X]=\int_{-A}^{\mu}x\mathrm{e}^{(\theta x -\theta \mu)} dx + \int_{\mu}^{A}x\mathrm{e}^{(-\theta x +\theta \mu)} dx \\ \nonumber
    =\left[\frac{(\theta x-1)\mathrm{e}^{\theta(x-\mu)}}{2\theta}\right]_{-A}^{\mu} + \left[-\frac{(\theta x+1)\mathrm{e}^{-\theta(x-\mu)}}{2\theta}\right]_{\mu}^{A} \\ \nonumber
    = \left(\frac{\theta \mu-1}{2\theta}\right) - \left(\frac{(-\theta A-1)\mathrm{e}^{-\theta(A+\mu)}}{2\theta}\right) + \left(-\frac{(\theta A+1)\mathrm{e}^{-\theta(A-\mu)}}{2\theta}\right) - \left(-\frac{\theta \mu+1}{2\theta}\right)
\end{eqnarray}
式(\ref{eq:12})について$A \to \infty$とすると,
\begin{equation}
    E_p[X] = \left(\frac{\theta \mu-1}{2\theta}\right) - 0 + 0 - \left(-\frac{\theta \mu+1}{2\theta}\right)
    = \mu \label{eq:13}
\end{equation}
$E_p[X] = \mu$となる.

次に$V_p[X]$について求める.
式(\ref{eq:3})に式(\ref{eq:13})を代入して計算すると
\begin{eqnarray}
    V_p[X]=\int_{-\infty}^{\infty}(x-E_p[X])^2p(x|\theta)dx \\ \nonumber
    = \int_{-\infty}^{\infty}x^2p(x|\theta) dx - 2^\mu\int_{-\infty}^{\infty}xp(x|\theta) dx + \mu^2\int_{-\infty}^{\infty}p(x|\theta) dx \label{eq:14}
\end{eqnarray}
確率密度関数$p(x|\theta)$を($-\infty, \infty$)まで広義積分すると
\begin{eqnarray}
    \int_{-\infty}^{\infty}p(x|\theta) dx = 1 \label{eq:15}
\end{eqnarray}
また,$\int_{-\infty}^{\infty}x^2p(x|\theta) dx$について部分積分を適用し計算すると
\begin{eqnarray}
    \int_{-\infty}^{\infty}x^2p(x|\theta) dx = \mu^2 + \frac{2}{\theta^2} \label{eq:16}
\end{eqnarray}
となる.
さらに
\begin{eqnarray}
    E_p[X] = \int_{-\infty}^{\infty}xp(x|\theta) dx = \mu \label{eq:17}
\end{eqnarray}
となるため,式(\ref{eq:14})に(\ref{eq:15}),(\ref{eq:16}),(\ref{eq:17})を代入し計算すると
\begin{eqnarray}
    V_p[X]=\int_{-\infty}^{\infty}x^2p(x|\theta) dx - 2^\mu\int_{-\infty}^{\infty}xp(x|\theta) dx + \mu^2\int_{-\infty}^{\infty}p(x|\theta) dx \\ \nonumber
    = \mu^2 + \frac{2}{\theta^2} - 2\mu^2 + \mu^2 = \frac{2}{\theta^2} \label{eq:18}
\end{eqnarray}
以上より
\begin{eqnarray}\label{eq:19}
    E_p[X] = \mu \\ \nonumber
    V_p[X] = \frac{2}{\theta}
\end{eqnarray}



% 数式(\ref{eq:5})の条件を数式(\ref{eq:4})に適用すると不定積分区間は次のように場合分けできる
% \begin{equation}
%     E_p[X]=\frac{\theta}{2}\int_{-\infty}^{\mu}x\exp(\theta(x-\mu))dx + \frac{\theta}{2}\int_{\mu}^{\infty}x\exp(-\theta(x-\mu))dx \label{eq:6}
% \end{equation}
% 被積分関数$x\exp(\theta(x-\mu)), x\exp(-\theta(x-\mu))$について,
% \begin{eqnarray} \label{eq:7}
%     s=\theta(x-\mu), ds =\theta dx \\ \nonumber
%     t=-\theta(x-\mu), dt=\theta dx \\ \nonumber
% \end{eqnarray}
% を置換すると
% \begin{eqnarray}
%     \frac{1}{2\theta}\int_{-\infty}^{\mu}exp(s)(s+\theta\mu)ds + \frac{1}{2\theta}\int_{\mu}^{\infty}exp(t)(t-\theta\mu)dt \label{eq:8}
% \end{eqnarray}
% 被積分関数$exp(s)(s+\theta\mu)$を展開し多項式の積分を各項に分けて定数を前に出すと
% \begin{eqnarray}
%     \frac{1}{2\theta}\int_{-\infty}^{\mu}sexp(s) ds + \frac{\mu}{2}\int_{-\infty}^{\mu} exp(s) ds + \frac{1}{2\theta}\int_{\mu}^{\infty}texp(t) dt - \frac{\mu}{2}\int_{\mu}^{\infty}exp(t) dt \label{eq:9}
% \end{eqnarray}
% 被積分関数$sexp(s), texp(t)$について,部分積分を適用すると,
% \begin{eqnarray}
%     [sexp(s)]_{-\infty}^{\mu} - \int_{-\infty}^{\mu}exp(s) ds + \frac{\mu}{2}\int_{-\infty}^{\mu} exp(s) ds + \\ \nonumber 
%     [texp(t)]_{\mu}^{\infty} - \int_{\mu}^{\infty}exp(t) dt - \frac{\mu}{2}\int_{\mu}^{\infty}exp(t) dt \label{eq:10}
% \end{eqnarray}
% 計算すると
% \begin{eqnarray}
%     \mu exp(\mu)
% \end{eqnarray}
\end{document}