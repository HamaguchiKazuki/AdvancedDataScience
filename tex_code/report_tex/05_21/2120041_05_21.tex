\documentclass[uplatex,b5j]{jsarticle} % uplatexオプションを入れる
\usepackage[scale=.8]{geometry}      % ページレイアウトを変更してみる
\usepackage[T1]{fontenc}             % T1エンコーディングにしてみる
\usepackage{txfonts}                 % 欧文フォントを変えてみる
\usepackage{plext}                   % pLaTeX付属の縦書き支援パッケージ
\usepackage{okumacro}                % jsclassesに同梱のパッケージ
\usepackage{amsmath}
\begin{document}
\title{データ科学特論 課題1}
\author{2120041 濱口和希}
\西暦\maketitle                      % 漢字のマクロ名もOK

\section{}
\subsection{a}
\begin{eqnarray}
    \frac{\partial L(\boldsymbol{\theta}, \lambda)}{\partial \boldsymbol{\theta}} =
    \sum_{i=1}^{2}\frac{\partial (-f(x_i;\theta))}{\partial \boldsymbol{\theta}}(y_i-(x_i,1)\boldsymbol{\theta})
    +\frac{\lambda}{2}\frac{\partial}{\partial \boldsymbol{\theta}}\boldsymbol{\theta}^T \boldsymbol{\theta} \\ \nonumber
    = -\sum_{i=1}^{2}(x_i, 1)^T(y_i-(x_i,1)\boldsymbol{\theta})+\lambda\boldsymbol{\theta} \\ \nonumber
    = -\sum_{i=1}^{2}(x_iy_i, y_i)^T+(\sum_{i=1}^{2}(x_i,1)^T(x_i,1) + \lambda)\boldsymbol{\theta} \\ \nonumber
    = \left(\begin{array}{c} 1+ab \\ 1+b \end{array} \right) + \left(\left(\begin{array}{cc} 1&1 \\ 1&1 \end{array} \right)
    + \left(\begin{array}{cc} a^2&a \\ a&1 \end{array} \right) + \left(\begin{array}{cc} \lambda&\lambda \\ \lambda&\lambda \end{array} \right)\right)\boldsymbol{\theta} = 0
\end{eqnarray}
したがって
\begin{eqnarray}
    \left(\begin{array}{cc} 1+a^2+\lambda & 1+a+\lambda \\ 1+a+\lambda & 1+\lambda \end{array} \right)\boldsymbol{\theta} = \left(\begin{array}{c} 1+ab \\ 1+b \end{array} \right),
    \boldsymbol{A} = \left(\begin{array}{cc} 1+a^2+\lambda & 1+a+\lambda \\ 1+a+\lambda & 1+\lambda \end{array} \right),
    \boldsymbol{B} = \left(\begin{array}{c} 1+ab \\ 1+b \end{array} \right)
\end{eqnarray}

\subsection{b}
\begin{eqnarray}
    \left| \begin{array}{cc} 1+a^2+\lambda & 1+a+\lambda \\ 1+a+\lambda & 1+\lambda \end{array} \right|
    = (1+a^2+\lambda)(2+\lambda)-(1+a+\lambda)^2 \\ \nonumber
    = a^2(\lambda+1)-2a(\lambda+1)+(\lambda+1) = (\lambda+1)(a-1)^2 \neq 0
\end{eqnarray}
したがって$a\neq1$の時解が存在する
\subsection{c}
$argmin\boldsymbol{\theta}_\lambda=\left( \begin{array}{c} \theta_1=0 \\ \theta_2=0 \end{array} \right)$となるような$\lambda$を解けばよい
\begin{eqnarray}
    \lambda = \left( \begin{array}{c} \frac{1}{b}-1 \\ \frac{a}{b}-1 \end{array} \right)
\end{eqnarray}
また
\begin{eqnarray}
    \lim_{\lambda\to\infty}\boldsymbol{\theta}_\lambda = \frac{b}{a-1} \left( \begin{array}{c} 1 \\ -1 \end{array} \right)
\end{eqnarray}

\section{}
$L(\boldsymbol{\theta}, \lambda)$を
\begin{eqnarray}
    L(\boldsymbol{\theta}, \lambda)
    =\frac{1}{2}(\boldsymbol{Y}-\boldsymbol{X\theta})^T(\boldsymbol{Y}-\boldsymbol{X\theta})+\frac{1}{2}\lambda\boldsymbol{\theta}^T\boldsymbol{\theta} \\ \nonumber
    (ただし \sum_{i=1}^{n}y_i=\boldsymbol{Y}, \sum_{i=1}^{n}f(x_i;\boldsymbol{\theta})=\boldsymbol{X\theta})
\end{eqnarray}
と書き換える.正規方程式で表したいため
\begin{eqnarray}
    \frac{\partial}{\partial \boldsymbol{\theta}}L(\boldsymbol{\theta}, \lambda) = 0 とすると \\ \nonumber
    \frac{\partial}{\partial \boldsymbol{\theta}}L(\boldsymbol{\theta}, \lambda)
    = -\boldsymbol{X}^T(\boldsymbol{Y}-\boldsymbol{X\theta})+\lambda\boldsymbol{\theta} = 0 \\ \nonumber
    \boldsymbol{X}^T\boldsymbol{X}\boldsymbol{\theta}+\lambda\boldsymbol{\theta} = \boldsymbol{X}^T\boldsymbol{Y} \\ \nonumber
    (\boldsymbol{X}^T\boldsymbol{X}+\lambda\boldsymbol{I})\boldsymbol{\theta} = \boldsymbol{X}^T\boldsymbol{Y} \\ \nonumber
    \boldsymbol{\theta}_\lambda = (\boldsymbol{X}^T\boldsymbol{X}+\lambda\boldsymbol{I})^{-1}\boldsymbol{X}^T\boldsymbol{Y}
\end{eqnarray}
したがってパラメータ$\boldsymbol{\theta}$は$(\boldsymbol{X}^T\boldsymbol{X}+\lambda\boldsymbol{I})^{-1}\boldsymbol{X}^T\boldsymbol{Y}$を満たす必要がある.



% $f(x_i;\theta)=\theta_1x_i+\theta_2$を式(\ref{eq:1})に代入すると
% \begin{eqnarray}
%     L(\theta) = \frac{1}{2}\sum_{i=1}^{2}(y_i - (\theta_1x_i+\theta_2))^2 = \frac{1}{2}\sum_{i=1}^{2}(y_i - \theta_1x_i - \theta_2)^2 \label{eq:2}
% \end{eqnarray}
% ここで式(\ref{eq:2})を$\theta_1, \theta_2$で偏微分すると
% \begin{eqnarray} \label{eq:3}
%     \left(
%     \begin{array}{c}
%         \frac{\partial L(\boldsymbol{\theta})}{\partial \theta_1} = 2\frac{1}{2}\sum_{i=1}^{2}(y_i - \theta_1x_i - \theta_2)(-x_i) = 0 \\
%         \frac{\partial L(\boldsymbol{\theta})}{\partial \theta_1} = 2\frac{1}{2}\sum_{i=1}^{2}(y_i - \theta_1x_i - \theta_2)(-1) = 0
%     \end{array}
%     \right)
% \end{eqnarray}
% 計算すると
% \begin{eqnarray}
%     \left(
%     \begin{array}{c} \label{eq:4}
%         \sum_{i=1}^{2}(y_i - \theta_1x_i - \theta_2)x_i = 0 \\
%         \sum_{i=1}^{2}(y_i - \theta_1x_i - \theta_2) = 0
%     \end{array}
%     \right) \\
%     =
%     \left(
%     \begin{array}{c} \label{eq:5}
%         (\sum_{i=1}^{2}x_{i}^{2})\theta_1 + (\sum_{i=1}^{2}x_i)\theta_2 = (\sum_{i=1}^{2}y_ix_i) \\
%         (\sum_{i=1}^{2}x_i)\theta_1 + (\sum_{i=1}^{2}1)\theta_2 = (\sum_{i=1}^{2}y_i)
%     \end{array}
%     \right)
% \end{eqnarray}
% ここで
% \begin{equation} \label{eq:6}
%     \boldsymbol{A}=\left[
%         \begin{array}{cc}
%             \sum_{i=1}^{2}x_{i}^{2} & \sum_{i=1}^{2}x_i \\
%             \sum_{i=1}^{2}x_i & \sum_{i=1}^{2}1
%         \end{array}
%     \right],
%     \boldsymbol{\theta}=\left[
%         \begin{array}{c}
%             \theta_1 \\
%             \theta_2
%         \end{array}
%     \right],
%     \boldsymbol{B}=\left[
%         \begin{array}{c}
%             \sum_{i=1}^{2}y_ix_i  \\
%             \sum_{i=1}^{2}y_i
%         \end{array}
%     \right]
% \end{equation}
% とすると$\boldsymbol{A\theta}=\boldsymbol{B}$に変形できる.
% 問題より$\mathcal{D}={(x_i, y_i)_{i=1}^2={(1,1),(a,b)}}$を与え$\boldsymbol{A}, \boldsymbol{B}$をそれぞれ計算すると
% \begin{eqnarray}
%     \boldsymbol{A}=\left[
%         \begin{array}{cc} \label{eq:7}
%             x_{1}^{2}+x_{2}^{2} & x_{1}+x_{2} \\
%             x_{1}+x_{2} & 2
%         \end{array}
%     \right]
%     = \left[
%         \begin{array}{cc}
%             1+a^{2} & 1+a \\
%             1+a & 2
%         \end{array}
%     \right]
%     \\
%     \boldsymbol{B}=\left[
%         \begin{array}{c} \label{eq:8}
%             x_{1}y_{1}+x_{2}y_{2}  \\
%             y_{1}+y_{2}
%         \end{array}
%     \right]
%     = \left[
%         \begin{array}{cc}
%             1+ab \\
%             1+b
%         \end{array}
%     \right]
% \end{eqnarray}
% 問1終了
% \section{}
% 問1より
% \begin{eqnarray} \label{eq:9}
%     \boldsymbol{A}=\left[
%         \begin{array}{cc}
%             1+a^{2} & 1+a \\
%             1+a & 2
%         \end{array}
%     \right]
% \end{eqnarray}
% 行列式$\boldsymbol{A}$の逆行列が存在するためには$\mathrm{det}\boldsymbol{A}\neq0$である必要がある.したがって,
% \begin{eqnarray} \label{eq:10}
%     \mathrm{det}\boldsymbol{A}=|\boldsymbol{A}|=\left|
%         \begin{array}{cc}
%             1+a^{2} & 1+a \\
%             1+a & 2
%         \end{array}
%     \right|
%     =2(1+a^2)-(1+a)(1+a)
%     =3a^2+2a+3\neq0
% \end{eqnarray}
% ここで,$a$を未知数として根を求めるため,解の公式を使用すると
% \begin{eqnarray} \label{eq:11}
%     a=\frac{-1-\sqrt{3\mathrm{j}}}{2}, \frac{-1+\sqrt{3\mathrm{j}}}{2}
% \end{eqnarray}
% となるため,$a\neq\frac{-1-\sqrt{3\mathrm{j}}}{2}, \frac{-1+\sqrt{3\mathrm{j}}}{2}(j=\sqrt(-1))$の時,行列式$\boldsymbol{A}$は逆行列を持つ正則行列となる.\\
% 問2終了
% \section{}
% 問1,2より行列式$\boldsymbol{A}$は$a\neq\frac{-1-\sqrt{3\mathrm{j}}}{2}, \frac{-1+\sqrt{3\mathrm{j}}}{2}$の時
% \begin{eqnarray} \label{eq:12}
%     \boldsymbol{\theta}=\boldsymbol{A^{-1}\boldsymbol{B}}
%     =\frac{1}{(a-1)^{2}}\left[
%         \begin{array}{cc}
%             2 & -(1+a) \\
%             -(1+a) & 1+a^{2}
%         \end{array}
%     \right]\left[
%         \begin{array}{c}
%             1 + ab \\
%             1 + b
%         \end{array}
%     \right]
% \end{eqnarray}
% 計算すると
% \begin{eqnarray} \label{eq:13}
%     \left[
%         \begin{array}{c}
%             \theta_1 \\
%             \theta_2
%         \end{array}
%     \right]
%     =\frac{1}{(a-1)^{2}}\left[
%         \begin{array}{c}
%             a^{2} - a + b \\
%             ab - a - b + 1
%         \end{array}
%     \right]
% \end{eqnarray}
% したがって,
% \begin{eqnarray}
%     \theta_1 = \frac{1}{(a-1)^{2}}(a^{2}-a+b) \\  \label{eq:14}
%     \theta_2 = \frac{1}{(a-1)^{2}}(ab-a-b+1)  \label{eq:15}
% \end{eqnarray}
% 問3終了
\end{document}