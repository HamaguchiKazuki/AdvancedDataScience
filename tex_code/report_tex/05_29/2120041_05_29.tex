\documentclass[uplatex,b5j]{jsarticle} % uplatexオプションを入れる
\usepackage[scale=.8]{geometry}      % ページレイアウトを変更してみる
\usepackage[T1]{fontenc}             % T1エンコーディングにしてみる
\usepackage{txfonts}                 % 欧文フォントを変えてみる
\usepackage{plext}                   % pLaTeX付属の縦書き支援パッケージ
\usepackage{okumacro}                % jsclassesに同梱のパッケージ
\usepackage{amsmath}
\begin{document}
\title{データ科学特論 課題}
\author{2120041 濱口和希}
\西暦\maketitle                      % 漢字のマクロ名もOK

\section{}
\subsection{a}
\subsection{b}


\section{}
\subsection{a}
まず$x, logx$の定義域を調べると
\begin{eqnarray}
    x \{x \in \mathbb{R}\} \\ \nonumber
    logx \{x \in \mathbb{R}, x > 0\} \\ \nonumber
    f(x) = x - logx \{x \in \mathbb{R}, x > 0\}
\end{eqnarray}
となる.ここで,極値を取るための必要条件として,$f^{(1)}(x)=0$を計算する.
\begin{eqnarray}
    f^{(1)}(x)=1-\frac{1}{x}=0 \\ \nonumber
    x = 1
\end{eqnarray}
$x=1$で極値を取るため$x > 0$における$f(x)$の増減表は表\ref{increase_decrease_matrix}になる.
\begin{table}
    \centering
    \caption{$f(x)$の増減表}
    \label{increase_decrease_matrix}
    \begin{tabular}{|c|c|c|c|c|c|}
    \hline
    $x$    & $0$   & $\cdots$ & $1$ & $\cdots$ & $\infty$ \\ \hline
    $f^{(1)}(x)$ &     & $-$   & $0$ & $+$   &     \\ \hline
    $f(x)$  & $\infty$ &  $\cdots$   &  $1$ &  $\cdots$   & $\infty$ \\ \hline
    \end{tabular}
\end{table}
したがって定義域$x>0$において$f(x)$の値域は$1 \leq f(x) \leq \infty$となるため最小値は$1$である.
\end{document}