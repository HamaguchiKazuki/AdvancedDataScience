\documentclass[uplatex,b5j]{jsarticle} % uplatexオプションを入れる
\usepackage[scale=.8]{geometry}      % ページレイアウトを変更してみる
\usepackage[T1]{fontenc}             % T1エンコーディングにしてみる
\usepackage{txfonts}                 % 欧文フォントを変えてみる
\usepackage{plext}                   % pLaTeX付属の縦書き支援パッケージ
\usepackage{okumacro}                % jsclassesに同梱のパッケージ
\usepackage{amsmath}
\begin{document}
\title{データ科学特論 課題1}
\author{2120041 濱口和希}
\西暦\maketitle                      % 漢字のマクロ名もOK

\section{}
まず,$\boldsymbol{A\theta}=\boldsymbol{B}$の形に変形する.
問題より
\begin{eqnarray}
    L(\theta) = \frac{1}{2}\sum_{i=1}^{2}(y_i - f(x_i;\theta))^2 \label{eq:1}
\end{eqnarray}
$f(x_i;\theta)=\theta_1x_i+\theta_2$を式(\ref{eq:1})に代入すると
\begin{eqnarray}
    L(\theta) = \frac{1}{2}\sum_{i=1}^{2}(y_i - (\theta_1x_i+\theta_2))^2 = \frac{1}{2}\sum_{i=1}^{2}(y_i - \theta_1x_i - \theta_2)^2 \label{eq:2}
\end{eqnarray}
ここで式(\ref{eq:2})を$\theta_1, \theta_2$で偏微分すると
\begin{eqnarray} \label{eq:3}
    \left(
    \begin{array}{c}
        \frac{\partial L(\boldsymbol{\theta})}{\partial \theta_1} = 2\frac{1}{2}\sum_{i=1}^{2}(y_i - \theta_1x_i - \theta_2)(-x_i) = 0 \\
        \frac{\partial L(\boldsymbol{\theta})}{\partial \theta_1} = 2\frac{1}{2}\sum_{i=1}^{2}(y_i - \theta_1x_i - \theta_2)(-1) = 0
    \end{array}
    \right)
\end{eqnarray}
計算すると
\begin{eqnarray}
    \left(
    \begin{array}{c} \label{eq:4}
        \sum_{i=1}^{2}(y_i - \theta_1x_i - \theta_2)x_i = 0 \\
        \sum_{i=1}^{2}(y_i - \theta_1x_i - \theta_2) = 0
    \end{array}
    \right) \\
    =
    \left(
    \begin{array}{c} \label{eq:5}
        (\sum_{i=1}^{2}x_{i}^{2})\theta_1 + (\sum_{i=1}^{2}x_i)\theta_2 = (\sum_{i=1}^{2}y_ix_i) \\
        (\sum_{i=1}^{2}x_i)\theta_1 + (\sum_{i=1}^{2}1)\theta_2 = (\sum_{i=1}^{2}y_i)
    \end{array}
    \right)
\end{eqnarray}
ここで
\begin{equation} \label{eq:6}
    \boldsymbol{A}=\left[
        \begin{array}{cc}
            \sum_{i=1}^{2}x_{i}^{2} & \sum_{i=1}^{2}x_i \\
            \sum_{i=1}^{2}x_i & \sum_{i=1}^{2}1
        \end{array}
    \right],
    \boldsymbol{\theta}=\left[
        \begin{array}{c}
            \theta_1 \\
            \theta_2
        \end{array}
    \right],
    \boldsymbol{B}=\left[
        \begin{array}{c}
            \sum_{i=1}^{2}y_ix_i  \\
            \sum_{i=1}^{2}y_i
        \end{array}
    \right]
\end{equation}
とすると$\boldsymbol{A\theta}=\boldsymbol{B}$に変形できる.
問題より$\mathcal{D}={(x_i, y_i)_{i=1}^2={(1,1),(a,b)}}$を与え$\boldsymbol{A}, \boldsymbol{B}$をそれぞれ計算すると
\begin{eqnarray}
    \boldsymbol{A}=\left[
        \begin{array}{cc} \label{eq:7}
            x_{1}^{2}+x_{2}^{2} & x_{1}+x_{2} \\
            x_{1}+x_{2} & 2
        \end{array}
    \right]
    = \left[
        \begin{array}{cc}
            1+a^{2} & 1+a \\
            1+a & 2
        \end{array}
    \right]
    \\
    \boldsymbol{B}=\left[
        \begin{array}{c} \label{eq:8}
            x_{1}y_{1}+x_{2}y_{2}  \\
            y_{1}+y_{2}
        \end{array}
    \right]
    = \left[
        \begin{array}{cc}
            1+ab \\
            1+b
        \end{array}
    \right]
\end{eqnarray}
問1終了
\section{}
問1より
\begin{eqnarray} \label{eq:9}
    \boldsymbol{A}=\left[
        \begin{array}{cc}
            1+a^{2} & 1+a \\
            1+a & 2
        \end{array}
    \right]
\end{eqnarray}
行列式$\boldsymbol{A}$の逆行列が存在するためには$\mathrm{det}\boldsymbol{A}\neq0$である必要がある.したがって,
\begin{eqnarray} \label{eq:10}
    \mathrm{det}\boldsymbol{A}=|\boldsymbol{A}|=\left|
        \begin{array}{cc}
            1+a^{2} & 1+a \\
            1+a & 2
        \end{array}
    \right|
    =2(1+a^2)-(1+a)(1+a)
    =3a^2+2a+3\neq0
\end{eqnarray}
ここで,$a$を未知数として根を求めるため,解の公式を使用すると
\begin{eqnarray} \label{eq:11}
    a=\frac{-1-\sqrt{3\mathrm{j}}}{2}, \frac{-1+\sqrt{3\mathrm{j}}}{2}
\end{eqnarray}
となるため,$a\neq\frac{-1-\sqrt{3\mathrm{j}}}{2}, \frac{-1+\sqrt{3\mathrm{j}}}{2}(j=\sqrt(-1))$の時,行列式$\boldsymbol{A}$は逆行列を持つ正則行列となる.\\
問2終了
\section{}
問1,2より行列式$\boldsymbol{A}$は$a\neq\frac{-1-\sqrt{3\mathrm{j}}}{2}, \frac{-1+\sqrt{3\mathrm{j}}}{2}$の時
\begin{eqnarray} \label{eq:12}
    \boldsymbol{\theta}=\boldsymbol{A^{-1}\boldsymbol{B}}
    =\frac{1}{(a-1)^{2}}\left[
        \begin{array}{cc}
            2 & -(1+a) \\
            -(1+a) & 1+a^{2}
        \end{array}
    \right]\left[
        \begin{array}{c}
            1 + ab \\
            1 + b
        \end{array}
    \right]
\end{eqnarray}
計算すると
\begin{eqnarray} \label{eq:13}
    \left[
        \begin{array}{c}
            \theta_1 \\
            \theta_2
        \end{array}
    \right]
    =\frac{1}{(a-1)^{2}}\left[
        \begin{array}{c}
            a^{2} - a + b \\
            ab - a - b + 1
        \end{array}
    \right]
\end{eqnarray}
したがって,
\begin{eqnarray}
    \theta_1 = \frac{1}{(a-1)^{2}}(a^{2}-a+b) \\  \label{eq:14}
    \theta_2 = \frac{1}{(a-1)^{2}}(ab-a-b+1)  \label{eq:15}
\end{eqnarray}
問3終了
\end{document}