\documentclass[uplatex,b5j]{jsarticle} % uplatexオプションを入れる
\usepackage[scale=.8]{geometry}      % ページレイアウトを変更してみる
\usepackage[T1]{fontenc}             % T1エンコーディングにしてみる
\usepackage{txfonts}                 % 欧文フォントを変えてみる
\usepackage{plext}                   % pLaTeX付属の縦書き支援パッケージ
\usepackage{okumacro}                % jsclassesに同梱のパッケージ
\usepackage{amsmath}
\begin{document}
\title{データ科学特論 課題1}
\author{2120041 濱口和希}
\西暦\maketitle                      % 漢字のマクロ名もOK

\section{}
まず,$\boldsymbol{A\theta}=\boldsymbol{B}$の形に変形する.
問題より
\begin{eqnarray}
    L(\theta) = \frac{1}{2}\sum_{i=1}^{2}(y_i - f(x_i;\theta))^2 \label{eq:1}
\end{eqnarray}
$f(x_i;\theta)=\theta_1x_i+\theta_2$を式(\ref{eq:1})に代入すると
\begin{eqnarray}
    L(\theta) = \frac{1}{2}\sum_{i=1}^{2}(y_i - (\theta_1x_i+\theta_2))^2 = \frac{1}{2}\sum_{i=1}^{2}(y_i - \theta_1x_i - \theta_2)^2 \label{eq:2}
\end{eqnarray}
ここで式(\ref{eq:2})を$\theta_1, \theta_2$で偏微分すると
\begin{eqnarray} \label{eq:3}
    \left(
    \begin{array}{c}
        \frac{\partial L(\boldsymbol{\theta})}{\partial \theta_1} = 2\frac{1}{2}\sum_{i=1}^{2}(y_i - \theta_1x_i - \theta_2)(-x_i) = 0 \\
        \frac{\partial L(\boldsymbol{\theta})}{\partial \theta_1} = 2\frac{1}{2}\sum_{i=1}^{2}(y_i - \theta_1x_i - \theta_2)(-1) = 0
    \end{array}
    \right)
\end{eqnarray}
計算すると
\begin{eqnarray}
    \left(
    \begin{array}{c} \label{eq:4}
        \sum_{i=1}^{2}(y_i - \theta_1x_i - \theta_2)x_i = 0 \\
        \sum_{i=1}^{2}(y_i - \theta_1x_i - \theta_2) = 0
    \end{array}
    \right) \\
    =
    \left(
    \begin{array}{c} \label{eq:5}
        (\sum_{i=1}^{2}x_{i}^{2})\theta_1 + (\sum_{i=1}^{2}x_i)\theta_2 = (\sum_{i=1}^{2}y_ix_i) \\
        (\sum_{i=1}^{2}x_i)\theta_1 + (\sum_{i=1}^{2}1)\theta_2 = (\sum_{i=1}^{2}y_i)
    \end{array}
    \right)
\end{eqnarray}
ここで
\begin{equation} \label{eq:6}
    \boldsymbol{A}=\left[
        \begin{array}{cc}
            \sum_{i=1}^{2}x_{i}^{2} & \sum_{i=1}^{2}x_i \\
            \sum_{i=1}^{2}x_i & \sum_{i=1}^{2}1
        \end{array}
    \right],
    \boldsymbol{\theta}=\left[
        \begin{array}{c}
            \theta_1 \\
            \theta_2
        \end{array}
    \right],
    \boldsymbol{B}=\left[
        \begin{array}{c}
            \sum_{i=1}^{2}y_ix_i  \\
            \sum_{i=1}^{2}y_i
        \end{array}
    \right]
\end{equation}
とすると$\boldsymbol{A\theta}=\boldsymbol{B}$に変形できる.
問題より$\mathcal{D}={(x_i, y_i)_{i=1}^2={(1,1),(a,b)}}$を与え$\boldsymbol{A}, \boldsymbol{B}$をそれぞれ計算すると
\begin{eqnarray}
    \boldsymbol{A}=\left[
        \begin{array}{cc} \label{eq:7}
            x_{1}^{2}+x_{2}^{2} & x_{1}+x_{2} \\
            x_{1}+x_{2} & 2
        \end{array}
    \right]
    = \left[
        \begin{array}{cc}
            1+a^{2} & 1+a \\
            1+a & 2
        \end{array}
    \right]
    \\
    \boldsymbol{B}=\left[
        \begin{array}{c} \label{eq:8}
            x_{1}y_{1}+x_{2}y_{2}  \\
            y_{1}+y_{2}
        \end{array}
    \right]
    = \left[
        \begin{array}{cc}
            1+ab \\
            1+b
        \end{array}
    \right]
\end{eqnarray}
問1終了
\section{}
問1より


% 数式(\ref{eq:5})の条件を数式(\ref{eq:4})に適用すると不定積分区間は次のように場合分けできる
% \begin{equation}
%     E_p[X]=\frac{\theta}{2}\int_{-\infty}^{\mu}x\exp(\theta(x-\mu))dx + \frac{\theta}{2}\int_{\mu}^{\infty}x\exp(-\theta(x-\mu))dx \label{eq:6}
% \end{equation}
% 被積分関数$x\exp(\theta(x-\mu)), x\exp(-\theta(x-\mu))$について,
% \begin{eqnarray} \label{eq:7}
%     s=\theta(x-\mu), ds =\theta dx \\ \nonumber
%     t=-\theta(x-\mu), dt=\theta dx \\ \nonumber
% \end{eqnarray}
% を置換すると
% \begin{eqnarray}
%     \frac{1}{2\theta}\int_{-\infty}^{\mu}exp(s)(s+\theta\mu)ds + \frac{1}{2\theta}\int_{\mu}^{\infty}exp(t)(t-\theta\mu)dt \label{eq:8}
% \end{eqnarray}
% 被積分関数$exp(s)(s+\theta\mu)$を展開し多項式の積分を各項に分けて定数を前に出すと
% \begin{eqnarray}
%     \frac{1}{2\theta}\int_{-\infty}^{\mu}sexp(s) ds + \frac{\mu}{2}\int_{-\infty}^{\mu} exp(s) ds + \frac{1}{2\theta}\int_{\mu}^{\infty}texp(t) dt - \frac{\mu}{2}\int_{\mu}^{\infty}exp(t) dt \label{eq:9}
% \end{eqnarray}
% 被積分関数$sexp(s), texp(t)$について,部分積分を適用すると,
% \begin{eqnarray}
%     [sexp(s)]_{-\infty}^{\mu} - \int_{-\infty}^{\mu}exp(s) ds + \frac{\mu}{2}\int_{-\infty}^{\mu} exp(s) ds + \\ \nonumber 
%     [texp(t)]_{\mu}^{\infty} - \int_{\mu}^{\infty}exp(t) dt - \frac{\mu}{2}\int_{\mu}^{\infty}exp(t) dt \label{eq:10}
% \end{eqnarray}
% 計算すると
% \begin{eqnarray}
%     \mu exp(\mu)
% \end{eqnarray}
\end{document}